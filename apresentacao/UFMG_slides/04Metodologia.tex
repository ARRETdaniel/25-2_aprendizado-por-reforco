\section{System Architecture \& Implementation}

\begin{frame}
    \frametitle{ROS 2 + CARLA Integration Architecture}

    \textbf{Challenge:} CARLA 0.9.16 has native ROS 2 for \emph{sensors only}, not vehicle control

    \vspace{0.3cm}

    \begin{center}
        \begin{tikzpicture}[
            node distance=1.2cm,
            block/.style={rectangle, draw, thick, text width=2.8cm, minimum height=1cm, text centered, rounded corners, font=\small},
            arrow/.style={->, >=Stealth, thick}
        ]
            % Training container
            \node[block, fill=blue!20] (training) {Training Container\\Ubuntu 22.04\\Python 3.10};

            % ROS 2 Humble
            \node[block, fill=green!20, right=of training] (ros2) {ROS 2 Humble\\Native rclpy};

            % Twist to Control
            \node[block, fill=orange!20, below=0.8cm of ros2] (twist) {Twist-to-Control\\Converter};

            % CARLA
            \node[block, fill=red!20, left=of twist] (carla) {CARLA 0.9.16\\Simulator};

            % Arrows
            \draw[arrow] (training) -- node[above, font=\tiny] {TD3 actions} (ros2);
            \draw[arrow] (ros2) -- node[right, font=\tiny, text width=1.5cm] {geometry\_msgs/\\Twist} (twist);
            \draw[arrow] (twist) -- node[below, font=\tiny] {VehicleControl} (carla);
            \draw[arrow, dashed] (carla) -- node[left, font=\tiny, text width=1.2cm] {Sensors\\(native)} (training);
        \end{tikzpicture}
    \end{center}

    \vspace{0.2cm}

    \textbf{Solution:} Standard ROS 2 messages + custom converter
    \begin{itemize}
        \item \textbf{geometry\_msgs/Twist:} Standard ROS 2 (no custom packages!)
        \item \textbf{630× faster} than docker-exec approach
        \item \textbf{2.3 GB smaller} image (no carla\_msgs build)
    \end{itemize}
\end{frame}

\begin{frame}
    \frametitle{CNN Feature Extractor Architecture}

    \textbf{Visual Processing Pipeline:}
    \begin{enumerate}
        \item Input: 4 stacked grayscale frames (84×84) $\rightarrow$ temporal dynamics
        \item NatureCNN architecture (from DQN paper):
        \begin{itemize}
            \item Conv1: 32 filters, 8×8 kernel, stride 4
            \item Conv2: 64 filters, 4×4 kernel, stride 2
            \item Conv3: 64 filters, 3×3 kernel, stride 1
            \item Fully connected: 512 units
        \end{itemize}
        \item Output: 512-dim feature vector
        \item Concatenate with kinematic features (53-dim)
    \end{enumerate}

    \vspace{0.3cm}

    \textbf{Why this design?}
    \begin{itemize}
        \item Proven in Atari DQN \cite{mnih2015humanlevel}
        \item Captures spatial hierarchies and motion patterns
        \item Used by \cite{elallid2023deep} for intersection navigation
    \end{itemize}
\end{frame}

\begin{frame}
    \frametitle{TD3 Network Architecture}

    \begin{center}
        \resizebox{0.95\textwidth}{!}{%
        \begin{tikzpicture}[
            node distance=0.6cm and 0.9cm,
            block/.style={rectangle, draw, thick, minimum height=0.7cm, minimum width=1.8cm, font=\small, align=center},
            cnn/.style={block, fill=blue!30},
            mlp/.style={block, fill=green!30},
            concat/.style={circle, draw, very thick, minimum size=0.55cm, font=\normalsize, fill=white},
            arrow/.style={->, >=Stealth, thick},
            feedback/.style={->, >=Stealth, thick, dashed, color=red!60}
        ]
            % Input (left side)
            \node[block, fill=orange!30] (image) at (0,0) {Images\\4×84×84};
            \node[block, fill=yellow!30] (kinematic) at (0,-3.0) {Kinematic\\+ Waypoints\\53-dim};

            % Actor Path (TOP)
            \node[cnn, right=1.0cm of image] (actor_cnn) {Actor CNN\\512-dim};
            \node[concat, right=0.8cm of actor_cnn] (concat_actor) {⊕};
            \node[mlp, right=0.8cm of concat_actor] (actor_mlp) {Actor MLP\\565→256→256};
            \node[block, fill=purple!30, right=0.7cm of actor_mlp] (actions) {Actions\\2-dim\\tanh};

            % Critic Path (BOTTOM)
            \node[cnn, right=1.0cm of kinematic] (critic_cnn) {Critic CNN\\512-dim};
            \node[concat, right=0.8cm of critic_cnn] (concat_critic) {⊕};
            \node[mlp, right=0.8cm of concat_critic] (critic_mlp1) {Critic Q₁\\567→256→256→1};
            \node[mlp, below=0.35cm of critic_mlp1] (critic_mlp2) {Critic Q₂\\567→256→256→1};

            % Image to CNNs (both receive same input)
            \draw[arrow, color=blue!70] (image) -- (actor_cnn);
            \draw[arrow, color=blue!70] (image.south) -- ++(0,-0.5) -| (critic_cnn.north);

            % CNN to Concatenation
            \draw[arrow, color=blue!70] (actor_cnn) -- (concat_actor);
            \draw[arrow, color=blue!70] (critic_cnn) -- (concat_critic);

            % Kinematic to Concatenation (both receive same input)
            \draw[arrow, color=orange!80] (kinematic.north) -- ++(0,0.5) -| (concat_actor.south);
            \draw[arrow, color=orange!80] (kinematic) -- (concat_critic);

            % After concatenation flow
            \draw[arrow, color=green!70] (concat_actor) -- (actor_mlp);
            \draw[arrow, color=green!70] (actor_mlp) -- (actions);
            \draw[arrow, color=green!70] (concat_critic) -- (critic_mlp1);
            \draw[arrow, color=green!70] (concat_critic) -- (critic_mlp2);

            % Feedback: Actions to Critics
            \draw[feedback] (actions.south) -- ++(0,-0.4) -| ([xshift=0.2cm]critic_mlp1.east) node[pos=0.2, right, font=\tiny] {a};
            \draw[feedback] (actions.south) -- ++(0,-0.65) -| ([xshift=0.2cm]critic_mlp2.east);

        \end{tikzpicture}}
    \end{center}

\end{frame}

\begin{frame}
    \frametitle{TD3 Network Architecture}

    \vspace{0.1cm}

    {\small \textbf{Data Flow:} Images → Separate CNNs (512) ⊕ Kinematic (53) → State (565) → Actor/Critics}

    {\small \textbf{Key Innovation:} Separate CNNs prevent gradient interference \cite{fujimoto2018addressing}}

    {\small \textbf{Training:} LR=1e-3, batch=256, γ=0.99, τ=0.005, policy delay=2}
\end{frame}

\begin{frame}
    \frametitle{TD3 Innovation: Clipped Double Q-Learning}

    \textbf{Problem:} Single-critic methods (DDPG) suffer from \emph{overestimation bias}
    \begin{itemize}
        \item Function approximation errors cause Q-values to be overestimated
        \item Actor exploits these overestimates $\rightarrow$ learns suboptimal policies
    \end{itemize}

    \vspace{0.3cm}

    \textbf{TD3 Solution:} Twin Critics with minimum operator

    \vspace{0.2cm}

    \begin{columns}
        \begin{column}{0.48\textwidth}
            \textbf{DDPG (Single Critic):}
            \begin{equation*}
                y = r + \gamma Q_{\theta'}(s', \mu_{\phi'}(s'))
            \end{equation*}
            \begin{itemize}
                \item[\textcolor{red}{✗}] Overestimates Q-values
                \item[\textcolor{red}{✗}] Unstable training
            \end{itemize}
        \end{column}
        \begin{column}{0.48\textwidth}
            \textbf{TD3 (Twin Critics):}
            \begin{equation*}
                y = r + \gamma \min(Q_{\theta_1'}(s', \tilde{a}), Q_{\theta_2'}(s', \tilde{a}))
            \end{equation*}
            \begin{itemize}
                \item[\textcolor{green}{✓}] Conservative estimate
                \item[\textcolor{green}{✓}] Stable convergence
            \end{itemize}
        \end{column}
    \end{columns}

\end{frame}


\begin{frame}
    \frametitle{TD3 Innovation: Clipped Double Q-Learning}

    \vspace{0.3cm}

    \textbf{How it works:}
    \begin{enumerate}
        \item Two independent Q-networks estimate $Q_1(s,a)$ and $Q_2(s,a)$
        \item Target computation uses \textbf{minimum}: $\min(Q_1', Q_2')$ (upper bound)
        \item Both critics train toward the \textbf{same conservative target}
        \item Actor maximizes $Q_1(s, \mu(s))$ for policy gradient
    \end{enumerate}

    \vspace{0.1cm}

    {\small \textbf{Result:} Prevents value overestimation, improves policy quality \cite{fujimoto2018addressing}}
\end{frame}

\begin{frame}
    \frametitle{CARLA Environment Configuration}

    \textbf{Simulation Setup:}
    \begin{itemize}
        \item CARLA 0.9.16, Town01 with a 222m pre-defined route waypoint path with turn and 20 NPC vehicles
        %\item Synchronous mode (50 Hz, $\Delta t = 0.05s$)
        %\item Ego vehicle: Tesla Model 3
        %\item Traffic: 20 NPC vehicles (scenario 0)
        %\item Route: 222m pre-defined waypoint path with turn
    \end{itemize}

    \vspace{0.1cm}

    \begin{figure}
        \centering
        \includegraphics[width=0.4\textwidth]{imagens/robotica-2025-2/town01.png}
        \caption{CARLA Town01 testing environment}
    \end{figure}

    \small{\textbf{Episode Termination:} Collision, off-road, waypoint completion, or 1000 steps}
\end{frame}
